\chapter{Summary and Conclusions}

Our senior design team sought to design and implement an unmanned vehicle that gathers and relays information about a post-fire environment back to its operator so that humans do not have to subject themselves to hazardous environments to do so. Our team identified five key subsystem that when combined, will create a fully functioning vehicle fit to meet our customer needs.  These subsystems are environment sensing, user interface, communications, power and localization.

Overall, we believe that our design is adequate to solve the problem, by providing the necessary tools to give operators the ability to investigate a post-fire environment without placing themselves in immediate danger. The sensor packages, including the air particulate sensors, gas sensors and cameras, will provide data on environmental hazards.  Our user-interface will make it easy for operators to both drive the vehicle and view the incoming data from these sensor packages so that operators can quickly identify potential hazards.

One area in which our design is lacking is in the communications subsystem.  Due to budget constraints, we are relying on a short range peer-to-peer wifi connection to communicate with the vehicle.  This works for prototyping the functions of the vehicle, but future teams should consider upgrading the communication subsystem to something that has a longer ranger and greater reliability.  

While not included in the scope of our project, future teams could make use of the localization subsystem and the Robotic Operating System (ROS) to develop autonomous functionality to run on the vehicle. All of our design decisions tried to keep this ultimate goal of full vehicle autonomy in mind. We are now the third team to have worked on this vehicle and we hope that future teams will be able to reuse and improve upon our design, similar to how we built upon the work of previous teams. We envision the next phase of work including developing semi-autonomous functionality, identifying obstacles and points of interest automatically and alerting the operators. Over the next few years, future senior design teams could feasibly implement autonomous functionality, incorporating path planning and navigation stacks to allow the vehicle to drive simple trajectories on it's own. Finally, we hope that the vehicle is developed to the point where it can independently serve exploratory functions, navigating complex environments autonomously, even in communication-denied environments.

%TODO BEEF UP CONCLUSION

%So far, we have repaired the vehicle since we received it in an inoperable state. We are still running into minor issues that have caused short-lived setbacks in implementation, but we hope to have the vehicle assessed by a professional mechanic. We have also created web-based vehicle controls. A user can control the vehicle with an Xbox controller through a web page. In addition, we have itegrated a high level operating system (ROS), mounted a fully operable LiDAR, and have improved the overall power architecture. These are critical preliminary steps that will hopefully make further implementation go smoothly. 