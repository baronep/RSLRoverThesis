\chapter{Subsystem: Communications} \label{chap:communications}

Our vehicle needs to be able to relay information back to its operator and receive commands from the operator.  In order to achieve this, we need to design a strong communication link between the vehicle and the operator.

We currently plan on using a peer-to-peer wifi network to communicate with the vehicle.  Peer-to-peer wifi networks are easy to setup and maintain and provide the basic functionality necessary to send commands to the rover and receive data from all of the rover's sensor packages.

The main problem with the peer-to-peer wifi network is that it has a small range.  The connection is not good at long distances, and the actual distance varies depending on if there are any objects in between the network nodes. We recognize that the shortcomings of a peer-to-peer network are unacceptable and unsafe for our application. However, due to budget constraints and the technology that we had readily available to us, we decided to use the peer-to-peer wifi network anyways.  This will be enough to develop a functioning prototype of the vehicle and implement all of the other functionality that we need to.

Future teams should look to update this subsystem and use a technology that has a greater range and reliability.