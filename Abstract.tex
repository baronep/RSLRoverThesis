\begin{abstract}
\thispagestyle{plain}
\pagenumbering{roman}
\setcounter{page}{2}
     The goal of this project is to design and implement an unmanned vehicle that can assess the air quality and general state of a post-fire environment. To do this, we will equip Santa Clara University's Polaris 6x6 Ranger with appropriate sensors and cameras to determine how safe the environment is for humans to enter. We will also use GPS and laser scans to generate a 3D map that operators can use to define certain zones as particularly dangerous. Finally, we will incorporate partially-autonomous sensing capabilities that will allow the operator to easily drive the vehicle. The result will be a rugged, advanced, and intuitive vehicle that can be used to protect fire responders from any lingering hazards during the investigation of a post-fire environment. This vehicle will be accompanied by a powerful operating system and localization techniques that will allow any future research groups to help this vehicle evolve into a fully autonomous system. 
     
     %/So far, the vehicle has been repaired and is now able to be driven nominally, as it was received in an inoperable state. Also, a web-based interface to drive the vehicle by remote control has been developed and an accompanying powerful robotics operating system has been integrated into the system. Finally, the vehicle can now generate a map of its surroundings using the LiDAR laser that is mounted on the vehicle's roll cage. This document provides all the work we have contributed, as well as references to the work done on this vehicle by previous Senior Design groups. 
 


%\textbf{\textcolor{red}{
%Notes by Pat:
%\begin{itemize}
%\item I think this should probably be in future tense as we  haven't done much of it yet.
%\item We probably shouldn't mention autonomy at all in the abstract yet. 
%\item I think we should begin with more on the problem (forest fires, maybe some stats on how harmful these situations can be) and then touch on how out product meets these needs. Probably less on the actual implementation of it
%\end{itemize}
%}}
 
\end{abstract}