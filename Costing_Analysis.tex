\chapter{Costing Analysis}
% analysis of overall prototype costs vs. budget, and, if appropriate, production costing estimates and evaluation %

As part of the design process, a detailed, itemized budget was assembled for the purchase of the components necessary to fulfill our engineering requirements (Appendix \ref{App:Budget}). The most expensive items that we purchased are the sensors, for environmental sensing and for localization. We were fortunate enough to be provided with 2 SICK laser range finders by our advisor (Dr. Christopher Kitts). These items alone would have exceeded our budget of \$2500. In addition to the sensors provided by Dr. Kitts, we calculated that the environmental sensing packages cost roughly \$200. The remainder of our funds went towards communication and interface equipment as well as towards repairing the vehicle. As the vehicle is indeed an experimental research platform and is highly modified from a stock state, we allotted a generous amount of our funding to go towards repairing, tuning and upgrading the vehicle.

The RSL Rover Vehicle is a prototype and a technology demonstrator. Our primary goal is to showcase the capabilities of a vehicle with environmental sensing capabilities, remote drive by wire functionality and data visualization packages. Both cost and time are constraints to our design. While we currently do not offer a market-ready solution, we have considered features beyond what we will be providing, which would have great value in a commercial product. For example, for a market vehicle, we would want to have the vehicle adhere to as many specifications of MIL-STD-810G, the US Military's environmental resistance testing specifications, as possible. Additionally, features such as heat shielding, LTE connectivity, advanced and high powered RF transmitters and autonomy would all be desired capabilities  for a market-ready vehicle, but are beyond our budget and time line.