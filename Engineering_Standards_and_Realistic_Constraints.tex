\chapter{Engineering Standards and Realistic Constraints}
\section{Ethics} Any ethical concerns are very important to our design of this vehicle. We aim to potentially save the lives of fire responders. This can only be achieved if our system displays true sensor data to the best of our abilities. If our vehicle provides false values, that could mean the environment is inaccurately assessed and people could potentially perish. There might be some legal repercussions to this. We must ensure that our readings of environmental factors like air quality and temperature are accurate. It may not completely be our responsibility since the sensors are to be purchased from other vendors (we are not making the sensors ourselves), but we must be sure to handle these sensors properly and provide the most accurate data to our ability. 

To ensure that our readings are correct, we will be equipping the vehicle with three redundant sensor packages. This is done to ensure that any anomalous readings can be verified or disregarded, according to what the other sensor packages show.

	The vehicle itself, as a tool, is not unethical. Since its application is so specific and it is clear that the vehicle is to be used in a post-fire situation, we do not foresee any scenario in which our vehicle can be used unethically. We must be “good” engineers in the sense that we must be honest about our testing, results, and errors so that any future developers may not assume that all systems are working properly if they, in reality, are not. 

\section{Health and Safety} Our vehicle is housed in a small garage, where it sits on top of a handful of jack stands. We have a formal safety document that has been reviewed by certain Santa Clara University employees that deal with safety concerns. The document contains certain rules for us engineers to follow. For example:	
\begin{itemize}
\item make sure that the garage is properly ventilated 
\item have a "kill-switch" functionality for the vehicle at all times 
\item take special precaution when working underneath the vehicle or in any internal electrical wiring
\item there should always be at least two people present in the garage when working on the vehicle. 
\end{itemize}
We have explicit guidelines and our own intuition and common sense to follow. If we work on our vehicle in accordance with the safety documentation, it should be safe to operate by other users. All design decisions were made with the existing safety documentation in mind. For example, the vehicle will contain two kill-switches (one in the front and one in the back) and the remote controls will also contain a kill-switch.

The actual operators of this vehicle will also have a document that outlines all safety concerns and rules associated with operation and maintenance of this vehicle. 


\section{Manufacturability} The previous electrical work done to the vehicle is not easy to reproduce. This electrical work supports the "unmanned" feature of this vehicle. It was the 2014 RSL Rover team that created the "drive-by-wire" system. \cite{rslrover2014} 

This year, with ease of production in mind, we have decided to make all sensor packages as modular as possible. We would like the environmental sensors to be easily incorporated into any vehicle. We have kept the environmental sensor processors and other components completely separate from the subsystem that drives the vehicle. So, ideally, the disaster response aspect of this vehicle's design can be simple to be reproduce onto already manufactured unmanned (or even autonomous) vehicles. 

We have also decided to integrate a robotic operating system (ROS) that is considered to be the industry standard at this time. That way, any perfective or corrective maintenance on the system software can be done with ease by a team of competent programmers. 

\section{Environment} Environmental impact is not much of a relevant concern for our design. Since the vehicle was inherited, we must work with what is available to us. The Polaris 6x6 Ranger that we are using will undergo inspection by a professional mechanic who can determine whether the vehicle emits the appropriate amount of exhaust into the environment. However, our fire response vehicle design is not affected by environmental concerns, since none of the features we will be incorporating will have a positive or adverse affect on the environment. 

\section{Society} The motivation behind our project as a whole is to benefit society by keeping fire responders out of harm's way. We recognized an issue in society (that fire responders are exposed to deadly conditions while investigating a fire) and now we aim to fix this issue. We considered societal repercussions while making our design decisions, especially the designs that pertain to the "unmanned" feature of our vehicle. We recognize that unmanned and autonomous vehicles are slowly making their way into daily life. \cite{autonomouscars} We also recognize that this transition is met with great opposition. With any new technological advancements that transform activities that are heavily integrated into daily life (such as driving), many members of society react with fear. This is why we have decided to keep the driving functionality as transparent as possible. This means that we will not abstract away any driving mechanisms, such as the gear stick. When the operator changes gears, the actual gear stick will move accordingly. Keeping the driving functionality as familiar as possible will increase our vehicle's chances of being embraced by society. 



